%==============================================================================
% Sjabloon onderzoeksvoorstel bachproef
%==============================================================================
% Gebaseerd op document class `hogent-article'
% zie <https://github.com/HoGentTIN/latex-hogent-article>

% Voor een voorstel in het Engels: voeg de documentclass-optie [english] toe.
% Let op: kan enkel na toestemming van de bachelorproefcoördinator!
\documentclass{hogent-article}

% Invoegen bibliografiebestand
\addbibresource{voorstel.bib}

% Informatie over de opleiding, het vak en soort opdracht
\studyprogramme{Professionele bachelor toegepaste informatica}
\course{Bachelorproef}
\assignmenttype{Onderzoeksvoorstel}
% Voor een voorstel in het Engels, haal de volgende 3 regels uit commentaar
% \studyprogramme{Bachelor of applied information technology}
% \course{Bachelor thesis}
% \assignmenttype{Research proposal}

\academicyear{2025-2026} % TODO: pas het academiejaar aan

% TODO: Werktitel
\title{Technische Compliance Architectuur voor CRA \& NIS2: Implementatie van Zero-Trust en Systeemhardening van Embedded Linux Edge-Systemen gebruikt in zorg IT-infrastructuur.}

% TODO: Studentnaam en emailadres invullen
\author{Martijn Panneels}
\email{martijn.panneels@student.hogent.be}


% TODO: Geef de co-promotor op
\supervisor[Co-promotor]{F. De Sweaf (Exensio, \href{mailto:frederik.desweaf@gmail.com}{frederik.desweaf@gmail.com})}

% Binnen welke specialisatierichting uit 3TI situeert dit onderzoek zich?
% Kies uit deze lijst:
%
% - Mobile \& Enterprise development
% - AI \& Data Engineering
% - Functional \& Business Analysis
% - System \& Network Administrator
% - Mainframe Expert
% - Als het onderzoek niet past binnen een van deze domeinen specifieer je deze
%   zelf
%
\specialisation{System \& Network Administrator}
\keywords{compliance, CRA, NIS2, configuration, medical sector}

\begin{document}
  % Hier schrijf je de samenvatting van je voorstel, als een doorlopende tekst van één paragraaf. Let op: dit is geen inleiding, maar een samenvattende tekst van heel je voorstel met inleiding 
  % (voorstelling, kaderen thema), probleemstelling en centrale onderzoeksvraag, onderzoeksdoelstelling (wat zie je als het concrete resultaat van je bachelorproef?), 
  % voorgestelde methodologie, verwachte resultaten en meerwaarde van dit onderzoek
  % (wat heeft de doelgroep aan het resultaat?).
\begin{abstract}

  Ziekenhuizen zetten steeds meer slimme toestellen in, zoals sensoren in patiëntenkamers.
  \textcolor{red}{Deze toestellen zijn kwetsbaar voor cyberaanvallen omdat ze netwerkverbonden zijn en gevoelige patiëntendata verwerken.} 
  De Cyber Resilience Act (CRA) en NIS2-richtlijn stellen strikte beveiligingseisen aan dergelijke systemen. 
  De centrale onderzoeksvraag van dit voorstel is:
  \textcolor{red}{Hoe kan een embedded Debian Linux-systeem van de zorg IT-infrastructuurleverancier Exensio Exensio binnen een zorgnetwerk zo worden geconfigureerd dat het voldoet aan de vereisten van de Cyber Resilience Act en NIS2-richtlijn?}
  Om deze vraag te beantwoorden, wordt er een mapping gemaakt van de relevante wettelijke vereisten naar technische configuratie-eisen. 
  Hierna worden die toegepast op een embedded Linux-systeem volgens het Zero Trust-principe. 
  Eens geconfigureerd wordt het toestel opgezet binnen een gesimuleerd zorgnetwerk en worden de prestaties ervan geëvalueerd.
  Dit gebeurt met focus op Debian system hardening en Zero Trust-principes, zonder de bestaande data-pipeline te verstoren.
  Het onderzoek zal leiden tot een technisch gedocumenteerde configuratie die effectief voldoet aan de wetgeving en geplaatst kan worden in de bestaande pipeline.
  De verwachte conclusie is dat deze configuratie een basis zal zijn voor andere infrastructuurleveranciers die soortgelijke, compliante systemen willen implementeren.

\end{abstract}

\tableofcontents

% De hoofdtekst van het voorstel zit in een apart bestand, zodat het makkelijk
% kan opgenomen worden in de bijlagen van de bachelorproef zelf.
%---------- Inleiding ---------------------------------------------------------

% TODO: Is dit voorstel gebaseerd op een paper van Research Methods die je
% vorig jaar hebt ingediend? Heb je daarbij eventueel samengewerkt met een
% andere student?
% Zo ja, haal dan de tekst hieronder uit commentaar en pas aan.

%\paragraph{Opmerking}

% Dit voorstel is gebaseerd op het onderzoeksvoorstel dat werd geschreven in het
% kader van het vak Research Methods dat ik (vorig/dit) academiejaar heb
% uitgewerkt (met medesturent VOORNAAM NAAM als mede-auteur).
% 
% Waarover zal je bachelorproef gaan? Introduceer het thema en zorg dat volgende zaken zeker duidelijk aanwezig zijn:

% \begin{itemize}
%   \item kaderen thema
%   \item de doelgroep
%   \item de probleemstelling en (centrale) onderzoeksvraag
%   \item de onderzoeksdoelstelling
% \end{itemize}

% Denk er aan: een typische bachelorproef is \textit{toegepast onderzoek}, wat betekent dat je start vanuit een concrete probleemsituatie in bedrijfscontext, een \textbf{casus}. Het is belangrijk om je onderwerp goed af te bakenen: je gaat voor die \textit{ene specifieke probleemsituatie} op zoek naar een goede oplossing, op basis van de huidige kennis in het vakgebied.

% De doelgroep moet ook concreet en duidelijk zijn, dus geen algemene of vaag gedefinieerde groepen zoals \emph{bedrijven}, \emph{developers}, \emph{Vlamingen}, enz. Je richt je in elk geval op it-professionals, een bachelorproef is geen populariserende tekst. Eén specifiek bedrijf (die te maken hebben met een concrete probleemsituatie) is dus beter dan \emph{bedrijven} in het algemeen.

% Formuleer duidelijk de onderzoeksvraag! De begeleiders lezen nog steeds te veel voorstellen waarin we geen onderzoeksvraag terugvinden.

% Schrijf ook iets over de doelstelling. Wat zie je als het concrete eindresultaat van je onderzoek, naast de uitgeschreven scriptie? Is het een proof-of-concept, een rapport met aanbevelingen, \ldots Met welk eindresultaat kan je je bachelorproef als een succes beschouwen?


\section{Inleiding}%
\label{sec:inleiding}

Ziekenhuizen maken steeds meer gebruik van slimme toestellen, zoals sensoren in patiëntenkamers \autocite{Abdulmalek2022}. 
Deze toestellen zijn vaak verbonden met het netwerk en vormen daardoor aantrekkelijke doelwitten voor cybercriminelen. 
Patiëntendata is erg gegeerd omdat het gevoelige persoonsgegevens bevat.

Het is daarom belangrijk dat leveranciers van IT‑infrastructuur deze toestellen grondig beveiligen. 
De Europese Unie heeft hiervoor een wettelijk kader uitgewerkt dat leveranciers verplicht om hun apparaten en infrastructuur te beveiligen.
De Cyber Resilience Act (CRA) \autocite{CRA2024} en de NIS2‑richtlijn \autocite{NIS2024} zijn twee belangrijke wetgevingen die de beveiliging van digitale producten reguleren.

Dit onderzoek richt zich op \textcolor{red}{hoe een embedded Linux systeem binnen een zorgnetwerk zo kan worden geconfigureerd dat het voldoet aan de vereisten van de Cyber Resilience Act en de NIS2‑richtlijn.} 
Het te configureren systeem moet ingeschakeld worden in een bestaande pipeline, zonder de werking ervan te verstoren.
Het onderzoek is specifiek gericht op een casus van \textcolor{red}{zorg IT‑infrastructuurleverancier} Exensio, maar de resultaten kunnen ook toegepast worden door andere leveranciers in deze sector.

%---------- Stand van zaken ---------------------------------------------------
% Hier beschrijf je de \emph{state-of-the-art} rondom je gekozen onderzoeksdomein, d.w.z.\ een inleidende, doorlopende tekst over het onderzoeksdomein van je bachelorproef. 
% Je steunt daarbij heel sterk op de professionele \emph{vakliteratuur}, en niet zozeer op populariserende teksten voor een breed publiek. 
% Wat is de huidige stand van zaken in dit domein, en wat zijn nog eventuele open vragen (die misschien de aanleiding waren tot je onderzoeksvraag!)?

% Je mag de titel van deze sectie ook aanpassen (literatuurstudie, stand van zaken, enz.). Zijn er al gelijkaardige onderzoeken gevoerd? 
% Wat concluderen ze? Wat is het verschil met jouw onderzoek?

% Verwijs bij elke introductie van een term of bewering over het domein naar de vakliteratuur, bijvoorbeeld~\autocite{Hykes2013}! 
% Denk zeker goed na welke werken je refereert en waarom.

% Draag zorg voor correcte literatuurverwijzingen! Een bronvermelding hoort thuis \emph{binnen} de zin waar je je op die bron baseert, dus niet er buiten! 
% Maak meteen een verwijzing als je gebruik maakt van een bron. Doe dit dus \emph{niet} aan het einde van een lange paragraaf. 
% Baseer nooit teveel aansluitende tekst op eenzelfde bron.

% Als je informatie over bronnen verzamelt in JabRef, zorg er dan voor dat alle nodige info aanwezig is om de bron terug te vinden (zoals uitvoerig besproken in de lessen Research Methods).


\section{Literatuurstudie}%
\label{sec:literatuurstudie}

\subsection{Wetgevend kader}
Ten eerste is het belangrijk om het wetgevend kader rond cybersecurity in Belgische en Europese context goed te begrijpen.

\subsubsection{Wat is MDR}
Voor dit deel wordt de Medical Device Regulation (MDR) meer in detail onderzocht \autocite{MDR2025}. 
Eerst wordt bepaald of het beoogde embedded Linux systeem kan worden beschouwd als een medisch hulpmiddel of een onderdeel daarvan, en dus binnen het toepassingsgebied van de MDR valt.
Als blijkt dat het systeem onder de MDR valt, worden de meest relevante artikelen besproken. 
Kijken naar CRA en NIS2 is dan niet nodig, aangezien de MDR strengere eisen stelt op het gebied van cyberbeveiliging.
Wanneer het systeem niet onder de MDR valt, zal besproken worden waarom en worden de 2 volgende secties verder uitgewerkt.
\subsubsection{Wat is CRA}
In dit onderdeel wordt de Cyber Resilience Act (CRA) besproken \autocite{CRA2024}. 
Er wordt extra duiding gegeven over welke artikels relevant zijn voor deze specifieke casus.

\subsubsection{Wat is NIS2}
In dit onderdeel wordt de NIS2-richtlijn besproken \autocite{NIS2025}. 
Er wordt extra duiding gegeven over welke artikels relevant zijn voor deze specifieke casus.

\subsection{Technische concepten}
Nadat het duidelijk is welke artikels van toepassing zijn, worden de technische concepten onderzocht die nodig zijn om aan deze vereisten te voldoen.
Het systeem is een x86-based Debian Linux systeem en moet een minimale userspace hebben. 
\subsubsection{Debian System Hardening}
Tijdens het configureren van het systeem zal er gebruik worden gemaakt van systeemhardening technieken specifiek voor Debian-gebaseerde systemen. 
De configuratie zal maximaal gebaseerd zijn op officiële documentatie om zo betrouwbaar mogelijk te zijn, bijvoorbeeld uit de Debian Administrator's Handbook \autocite{Hertzog2020}.
\subsubsection{Zero-Trust best practices}
Om de Zero-Trust architectuur en mentaliteit te implementeren, worden er best practices onderzocht \autocite{Reddy2025} en \autocite{NIST2025}.   


% 
% Voor literatuurverwijzingen zijn er twee belangrijke commando's:
% \autocite{KEY} => (Auteur, jaartal) Gebruik dit als de naam van de auteur
%   geen onderdeel is van de zin.
% \textcite{KEY} => Auteur (jaartal)  Gebruik dit als de auteursnaam wel een
%   functie heeft in de zin (bv. ``Uit onderzoek door Doll & Hill (1954) bleek
%   ...'')

% Je mag deze sectie nog verder onderverdelen in subsecties als dit de structuur van de tekst kan verduidelijken.

%---------- Methodologie ------------------------------------------------------

% Hier beschrijf je hoe je van plan bent het onderzoek te voeren. Welke onderzoekstechniek ga je toepassen om elk van je onderzoeksvragen te beantwoorden? Gebruik je hiervoor literatuurstudie, interviews met belanghebbenden (bv.~voor requirements-analyse), experimenten, simulaties, vergelijkende studie, risico-analyse, PoC, \ldots?

% Valt je onderwerp onder één van de typische soorten bachelorproeven die besproken zijn in de lessen Research Methods (bv.\ vergelijkende studie of risico-analyse)? Zorg er dan ook voor dat we duidelijk de verschillende stappen terug vinden die we verwachten in dit soort onderzoek!

% Vermijd onderzoekstechnieken die geen objectieve, meetbare resultaten kunnen opleveren. Enquêtes, bijvoorbeeld, zijn voor een bachelorproef informatica meestal \textbf{niet geschikt}. De antwoorden zijn eerder meningen dan feiten en in de praktijk blijkt het ook bijzonder moeilijk om voldoende respondenten te vinden. Studenten die een enquête willen voeren, hebben meestal ook geen goede definitie van de populatie, waardoor ook niet kan aangetoond worden dat eventuele resultaten representatief zijn.

% Uit dit onderdeel moet duidelijk naar voor komen dat je bachelorproef ook technisch voldoen\-de diepgang zal bevatten. Het zou niet kloppen als een bachelorproef informatica ook door bv.\ een student marketing zou kunnen uitgevoerd worden.

% Je beschrijft ook al welke tools (hardware, software, diensten, \ldots) je denkt hiervoor te gebruiken of te ontwikkelen.

% Probeer ook een tijdschatting te maken. Hoe lang zal je met elke fase van je onderzoek bezig zijn en wat zijn de concrete \emph{deliverables} in elke fase?

\section{Methodologie}%
\label{sec:methodologie}

\subsection{Literatuurstudie}

  In de literatuurstudie zal er uitgebreid worden onderzocht welke wet-artikels uit CRA en NIS2 betrekking hebben op deze specifieke use-case. 
  Tegelijkertijd worden best practices voor Zero-Trust architectuur, systeemhardening en de beveiliging van real-time data pipelines op embedded Linux-systemen opgezocht. 
  De literatuurstudie zal doorheen het gehele project lopen en zal vermoedelijk 5 weken in beslag nemen.

\subsection{Verwerken tot requirements}

  In deze fase worden de bevindingen van de literatuurstudie gemapt naar een concreet technisch plan. 
  Dit plan zal de specifieke configuratie-eisen bevatten waaraan het Linux systeem moet voldoen om te voldoen aan CRA en NIS2. 
  Ook worden de zero-trust principes en systeemhardening technieken opgesteld die toepasbaar zijn voor dit systeem.
  Dit omvat het technisch ontwerp van hoe het systeem zal geconfigureerd worden. Deze fase zal ongeveer 3 weken duren.

\subsection{Requirements configureren}

  Tijdens de implementatie fase zal het plan uit de vorige fase worden omgezet naar een daadwerkelijke configuratie van het Linux systeem.  
  Dit omvat het implementeren van zero-trust principes en systeemhardening technieken compliant aan CRA en NIS2. 
  Het proof of concept zal worden opgezet in een gesimuleerd zorgnetwerk zodat het zo goed mogelijk de werkelijkheid nabootst. 
  Deze fase zal ongeveer 4 tot 5 weken duren.

\subsection{Configuratie testen}  

  In deze fase zal het proof of concept worden getest om te verzekeren dat het voldoet aan de vooropgestelde requirements. 
  De pentesting tool die hiervoor gebruikt zal worden wordt ter beschikking gesteld door Exensio. De resultaten van de pentest zullen gedocumenteerd worden. 
  Deze fase zal ongeveer 1 week duren.

\subsection{Configuratie evalueren}  

  In de laatste fase zal het geconfigureerde systeem worden geëvalueerd op basis van de testresultaten.
  Eventuele tekortkomingen zullen worden geïdentificeerd en er zullen aanbevelingen worden gedaan voor verdere verbeteringen.
  Deze fase zal ongeveer 1 week duren.




%---------- Verwachte resultaten ----------------------------------------------

% Hier beschrijf je welke resultaten je verwacht. Als je metingen en simulaties uitvoert, kan je hier al mock-ups maken van de grafieken samen met de verwachte conclusies. Benoem zeker al je assen en de onderdelen van de grafiek die je gaat gebruiken. Dit zorgt ervoor dat je concreet weet welk soort data je moet verzamelen en hoe je die moet meten.

% Wat heeft de doelgroep van je onderzoek aan het resultaat? Op welke manier zorgt jouw bachelorproef voor een meerwaarde?

% Hier beschrijf je wat je verwacht uit je onderzoek, met de motivatie waarom. Het is \textbf{niet} erg indien uit je onderzoek andere resultaten en conclusies vloeien dan dat je hier beschrijft: het is dan juist interessant om te onderzoeken waarom jouw hypothesen niet overeenkomen met de resultaten.

\section{Verwacht resultaat, conclusie}%
\label{sec:verwachte_resultaten}

Het voornaamste resultaat van dit onderzoek is een technisch gedocumenteerde configuratie voor het specifieke Linux systeem dat voldoet aan de vereisten van CRA en NIS2. 
Dit onderzoek zal echter een basis zijn voor andere infrastructuurleveranciers die soortgelijke systemen willen implementeren die compliant moeten zijn met deze regelgeving.

\printbibliography[heading=bibintoc]

\end{document}